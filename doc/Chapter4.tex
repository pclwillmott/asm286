% Activate the following line by filling in the right side. If for example the name of the root file is Main.tex, write
% "...root = Main.tex" if the chapter file is in the same directory, and "...root = ../Main.tex" if the chapter is in a subdirectory.
 
%!TEX root =  

\chapter[Defining And Initializing Data]{Defining And Initializing Data}

This chapter has four major sections:
\begin{itemize}
\item An overview of assembler labels, variables, and data
This section explains:
\begin{itemize}
\item Assembler label and variable types
\item The relationship between assembler variable types and the values associated with variables: the processor or floating-point coprocessor data types
\item How to specify data values in assembler programs
\end{itemize}
\item Assembler variables
This section explains:
\begin{itemize}
\item Storage allocations for variables
\item V ariable attributes
\item Defining and initializing simple-type variables with the DBIT, DB, DW, DD, DP, DQ, and DT directives
\item Defining compound types with the RECORD and STRUC directives; defining and initializing variables of these types (records and structures)
\item Defining and initializing variables with DUP clause(s)
\end{itemize}
\item Assembler labels
This section explains:
\begin{itemize}
\item Label attributes
\item The location counter and the ORG and EVEN directives
\item The LABEL directive
\item Defining implicit NEAR labels
\item The PROC directive
\end{itemize}
\item Using symbolic data, including named variables and labels, with the EQU and PURGE directives
\end{itemize}

\subsection*{Specifying Assembler Data Values}
Assembler data can be expressed in binary, hexadecimal, octal, decimal, or ASCII form. Decimal values that represent integers or reals can be specified with a minus sign; a plus sign is redundant but accepted. Real numbers can also be expressed in floating-point decimal or in hexadecimal notations. Table 4-2 summarizes the valid ways of specifying data values in assembler programs.

\begin{center}
Table 4-2. Assembler Data Value Specification Rules

\begin{tabular}{| l  l  l  p{6.5cm} |}
\hline
\textbf{Value in} & \textbf{Examples} & & \textbf{Rules of Formulation} \\ 
\hline
Binary & 1100011B & 110B & A sequence of 0's and 1's followed by the letter B. \\
& & & \\
Octal & 7777O & 4567Q & A sequence of digits in the range 0..7 followed by the letter O or the letter Q.\\
& & & \\
Decimal & 3309 & 3309D & A sequence of digits in the range 0..9 followed by an optional letter D.\\
& & & \\
Hexadecimal & 55H & 4BEACH & A sequence of digits in the range 0..9 and/or letters A..F followed by the letter H. A digit must begin the sequence.\\
& & & \\
ASCII & 'AB' & 'UPDATE.EXT' & Any ASCII string enclosed in single quotes.\\
& & & \\
Decimal & -1. & 1E-32 3.14159 & A rational number that may be preceded by a sign and followed by an optional exponent. A decimal point is required if no exponent is present but is optional otherwise. The exponent begins with the letter E followed by an optional sign and a sequence of digits in the range 0..9. \\ 
& & & \\
Hexadecimal & 40490FR & 0C0000R & A sequence of digits in the range 0..9 and/or letters A..F followed by the letter R. The sequence must begin with a digit, and the total number of digits and letters must be (8/16/20) or (9/17/21 with the first digit 0).
 \\
\hline
\end{tabular}
\end{center}
 
A real hexadecimal specification must be the exact sequence of hex digits to fill the internal floating-point coprocessor representation of the floating-point number. For this reason, such values must have exactly 8, 16, or 20 hexadecimal digits, corresponding to the single, double, and extended precision reals that the floating- point coprocessor and the floating-point instructions handle. Such values can have 9, 17, or 21 hexadecimal digits only if the initial digit must be a zero because the value begins with a letter.

Data values can be specified in an assembler program in a variety of formats, as shown in Table 4-2. The way the processor or floating-point coprocessor represents such data internally is called its storage format.

See also: Processor storage formats, Appendix A floating-point coprocessor storage formats, Chapter 7

\subsection*{Initializing Variables}
Assembler variables can be initialized by:
\begin{itemize}
\item Variable or segment names that represent logical addresses
\item Constants (see Table 4-2)
\item Constant expressions
\end{itemize}

A series of operands and operators is called an expression. An expression that yields a constant value is called a constant expression.

See also: Assembler expressions, Chapter 5

The assembler evaluates constant expressions in programs.

\subsection*{How the Assembler Evaluates Constant Expressions}
The assembler can perform arithmetic operations on 8-, and 16-bit numbers. The assembler interprets these numbers as integer or ordinal data types.

An integer value specified with a sign is a constant expression. The assembler evaluates integer or ordinal operands and expressions using 32-bit two's complement integer arithmetic.

By using this arithmetic, the assembler can evaluate expressions whose operands' sizes might extend beyond the storage type of the result. As long as the expression's value fits in the storage type of the destination, the assembler does not generate an error when intermediate results are too large. The assembler does generate an error if the final result is too large to fit in the destination.

\subsection*{Variables}
A variable defines a logical address for the storage of value(s). An assembler variable is not required to have a name as long as its associated value(s) are accessible. But, every variable has a type; records and structures have a compound type.

Assembler variables must be defined with a storage allocation statement. A storage allocation specifies a type (storage size in bytes) and defines a logical address for a variable that gives access to the variable's value(s). A storage allocation statement may also specify initial value(s) for a variable.

Use the DB, DW, DD, DP, DQ, or DT directive to allocate storage for simple-type variables of the following sizes:

\begin{tabular}{l l}
DB & 8-bits (byte)\\
DW & 16-bits (word)\\
DD & 32-bits (dword)\\
DP & 48-bits (pword)\\
DQ & 64-bits (qword)\\
DT & 80-bits (10 bytes)\\
\end{tabular}

Use a DUP clause within any assembler data allocation statement to allocate and optionally initialize a sequence of storage units of a single variable type. DUP defines an array-like variable whose element values are accessed by an offset from the variable name or from the initially specified storage unit.

\subsection*{Syntax}

\begin{tabular}{p{2cm} p{12.5cm}}
& \begin{verbatim}[name] dtyp init [,...]\end{verbatim}\\
Where: & \\
& \\
name & is the name of the variable. Within the module, it must be a unique identifier.\\
& \\
dtyp & is DB, DW, DD, DP, DQ, or DT.\\
& \\
init & is the initial value to be stored in the allocated space. init can be a numeric constant (expressed in binary, hexadecimal, decimal, or octal), an ASCII string, or the question mark character (?), which specifies storage with undefined value(s). dtyp restricts the values that may be specified for init.\\
\end{tabular}

\subsection*{Defining and Initializing Variables of a Simple Type}
All assembler variable definitions use the DB, DW, DD, DQ, DP, or DT directives. The template components of compound variable types are simple types defined with these directives.

\subsection*{DB Directive}
\subsection*{Syntax}
\begin{tabular}{p{2cm} p{12.5cm}}
& \begin{verbatim} [name] DB init [,...] \end{verbatim} \\
Where: & \\
& \\
name & is the name of the variable. Within the module, it must be a unique identifier. \\
& \\
init & is a question mark (?), a constant expression, or a string of up to 255 ASCII characters enclosed in single quotes (').\\
\end{tabular}
\subsection*{Discussion}
DB reserves storage for and optionally initializes a variable of type BYTE. ? reserves storage with an undefined value.

Numeric initial values can be specified in binary, octal, decimal, or hexadecimal (see Table 4-2). The specified constant or constant expression must evaluate to a number in the range 0..255 (processor ordinal) or -128..127 (processor integer).

The components of character string values must be ASCII characters and the whole string must be enclosed in single quotes. To include a single quote character within such a string, specify two single quotes ('').

Each ASCII character requires a byte of storage. In BYTE strings, successive characters occupy successive bytes. The name of the variable represents the logical address of the first character in such a string.

\subsection*{Examples}
\begin{enumerate}
\item This example initializes the variable ABYTE to the constant value 100 (decimal). It reserves storage for another byte with an undefined value.
\begin{verbatim}
 ABYTE DB 100
 DB ?
\end{verbatim}
\item This example initializes three successive bytes to the values 4, 10, and 200, respectively.
\begin{verbatim}
BYTES3 DB 4,10,200
\end{verbatim}
\item This example initializes seven bytes containing the ASCII values of the characters A, B, C, ' , D, E, and F, respectively.
\begin{verbatim}
STRGWQUOT DB 'ABC''DEF'
\end{verbatim}
\end{enumerate}
\subsection*{DW Directive} 
\subsection*{Syntax}

\begin{tabular}{p{2cm} p{12.5cm}}
& \begin{verbatim} [name] DW init [,...] \end{verbatim} \\
Where: & \\
& \\
name & is the name of the variable. Within the module, it must be a unique identifier. \\
& \\
init & is a question mark (?), a constant expression, or a string of up to 2 characters enclosed in single quotes (').\\
\end{tabular}
\subsection*{Discussion}
DW defines storage for and optionally initializes a 16-bit variable of type WORD. ? reserves storage with an undefined value.

Numeric initial values can be specified in binary, octal, decimal, or hexadecimal (see Table 4-2). The specified constant or constant expression must evaluate to a number in the range 0..65535 (processor ordinal) or -32768..32767 (processor integer).

A variable or label name yields an initial value that is the offset of the variable or label. It is an error to initialize a WORD variable with the name of a variable or label that has been defined in a USE32 segment; its offset is too large (32-bits). A segment name yields an initial value that is the segment selector.

A 1- or 2-character string yields an initial value that is interpreted and stored as a number. The assembler stores a 2-byte value even if the specified string has only one character:

\begin{itemize}
\item It stores the specified initial value in the least significant byte. 
\item It zeros the remaining byte.
\end{itemize}
\subsection*{Examples}
\begin{enumerate}
\item This example tells the assembler to reserve storage for two uninitialized words.
\begin{verbatim}
UNINIT DW ?,?
\end{verbatim}
\item This example initializes WORD variables with numeric values.
\begin{verbatim}
CONST DW 5000 ; decimal constant 
HEXEXP DW OFFFH -10 ; expression
\end{verbatim}
\item This example initializes VAR1OFF to the offset of VAR1 (both variables are within a USE16 segment) and CODESEL to the selector of a segment named CODE.
\begin{verbatim}
   VAR1OFF DW VAR1
   CODESEL DW CODE
   \end{verbatim}
\item This example initializes NUMB to the ASCII value (interpreted as a number) of the letters AB.
\begin{verbatim}
NUMB DW 'AB' ; equivalent to NUMB DW 4142H
\end{verbatim}
\end{enumerate}

\subsection*{DD Directive} 

\begin{tabular}{p{2cm} p{12.5cm}}
& \begin{verbatim} [name] DW init [,...] \end{verbatim} \\
Where: & \\
& \\
name & is the name of the variable. Within the module, it must be a unique identifier. \\
& \\
init & is a question mark (?), a constant expression, the name of a variable or label, or a string of up to 4 characters enclosed in single quotes (').\\
\end{tabular}

\subsection*{Discussion}
DD defines storage for and optionally initializes a 32-bit variable of type DWORD. ? reserves storage with an undefined value.

Integer initial values can be specified in binary, octal, decimal, or hexadecimal (see Table 4-2). The specified constant or constant expression must evaluate to a number in the range:
$-2^{31}$..$2^{31-1}$ (processor integer or floating-point coprocessor short integer) Or, 0..$2^{32-1}$ (processor ordinal)

Real initial values can be specified in floating-point decimal or in hexadecimal (see Table 4-2). A decimal constant must evaluate to a real in the ranges:
-3.4E38..-1.2E-38, 0.0, 1.2E-38..3.4E38 (floating-point coprocessor single precision real)

A constant expressed as a hexadecimal real must be the exact sequence of hex digits to fill the internal floating-point coprocessor representation of a single precision real (8 hexadecimal digits or 9 hexadecimal digits, including an initial 0).

A USE16 variable or label name yields an initial value that fills the dword. Its high-order word contains the segment selector and its low-order word contains the offset of the USE16 variable or label.

A USE32 variable or label name yields an initial value that is the offset (from the segment base) of the variable or label.

A string (up to four characters) yields an initial value that is interpreted and stored as a number. The assembler stores a 4-byte value even if the specified string has fewer than four characters: 
\begin{itemize}
\item It stores the specified initial values in the least significant bytes. 
\item It zeros the remaining bytes.
\end{itemize}

\subsection*{Examples}
\begin{enumerate}
\item This example defines two variables, a floating-point coprocessor short integer and a single precision real.
\begin{verbatim}
   INTVAR DD 1234567
   REALVAR DD 1.6E25
\end{verbatim}
\item In this example, LAB1 was defined in a USE16 segment and LAB2 was defined in a USE32 segment.
\begin{verbatim}
LAB1_ADD DD LAB1 ; LAB1_ADD contains offset and ; segment selector of LAB1
LAB2_ADD DD LAB2 ; LAB2_ADD contains offset of LAB2
\end{verbatim}
\item This example initializes three unnamed dwords. The first contains an undefined value. The second contains the ASCII numeric value of the letter A. The third contains the integer 450 (decimal).
\begin{verbatim}
DD ?, 'A', 450
\end{verbatim}
\end{enumerate}